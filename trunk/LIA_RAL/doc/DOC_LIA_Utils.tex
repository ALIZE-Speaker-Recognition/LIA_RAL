\documentclass[a4paper]{article}
\usepackage{DOC_LIA_RAL}
\usepackage{amssymb}
\usepackage[dvips]{epsfig}
\usepackage[dvips]{epsfig}

\date{\today}
\title{\Large\bf LIA\_SpkDet Package Documentation}
\name{LIA Automatic Speaker Recognition Team}
\address{LIA/CERI Universit\'e d'Avignon, Agroparc,\\ BP 1228, 84911
Avignon Cedex 9, France\\
\{lia@lia.univ-avignon.fr\}}

\begin{document}
\ninept
\maketitle
\href{http://www.lia.univ-avignon.fr}{\includegraphics[scale=0.8]{lia.eps}}\\

\section{Introduction to LIA\_RAL}

This package aims at providing Automatic Speaker Detection related
programs based on the ALIZE toolkit. It contains two task-specific
sub-packages: LIA\_SpkDet related to Speaker Detection and LIA\_Seg
related to Speaker Diarization and acoustic segmentation. A library
containing useful methods is provided in LIA\_SpkTools as well as
useful, uncategorized programs in LIA\_Utils. This documentation is
dedicated to the utilities package, LIA\_Utils
\section{LIA\_Utils}
\label{sec:Utils}

\subsection{Note}
Most of these programs tends to be useful, practical, quick to use, utilities. Indeed, no configuration files is required form most of them, so the user can quickly execute the program.
To prevent the user to give each time all the ALIZE options to program (e.g.: loadFeatureFileExtension, mixtureFilesPath, ...), we invite you to force some of the options when giving a model or a feature to the program by giving the path in the input Feature/Model File and the associate extension.\\
Example:
Instead of typing, ExtractParams.exe --inputFeatureFile test1 --loadFeatureFileExtension .prm --featureFilesPath ./, type  ExtractParams.exe --inputFeatureFile ./test1.prm.

\subsection{ExtractParams}

\subsubsection{Features}
ExtractParams aims to extract a subset of parameters in a feature file, thanks to the featureServerMask of ALIZE.

\subsubsection{Compulsory options}

\begin{tabular}{|c|c||p{8cm}|}
\hline Name & Example & Description\\
\hline
\hline inputFeatureFile & ./test1.prm & Name of the features to work with, can be a list with a .lst extension\\
\hline featureServerMask & "0-2,16-18" & Index of parameters to extract\\
\hline saveFeatureFileExtension & .norm.prm & Save extension\\
\hline
\end{tabular}

\subsection{FusionScore}

\subsubsection{Features}
FusionScore aims to implements fusion methods of score files. Method provided are arithmetic mean and geometric mean but there is no doubt a lot more will be available in the future.
It takes a list of score files as input and compute the fusion according to file giving the weights and an optional nBest number.

\subsubsection{Compulsory options}

\begin{tabular}{|c|c||p{8cm}|}
\hline Name & Example & Description\\
\hline
\hline inputFileList & foo.lst & Specify the list of input score files\\
\hline outputFile & foo.res & the resulting score file\\
\hline weights & myfusion.weights & coeffs of each file in a ascii file separate by a white space\\
\hline FusionMethods & ArithMean & <ArithMean|GeoMean>\\
\hline Nbest & 2 & optional, takes the nbest scores for the fusion\\
\hline
\end{tabular}

\subsection{Hist}
\subsubsection{Features}
Plot an Histogram where the width of bins are non equals (integral is equal to one). It is useful to draw a distribution.
Output is formatted so it can be directly be properly loaded and plotted in gnuplot. 
Example:
    If the output of Hist is foo.hist, in a shell, and considering you have gnuplot installed do:
    \$gnuplot
    \$gnuplot> plot "foo.hist" with boxes
 
\subsubsection{Compulsory options}

\begin{tabular}{|c|c||p{8cm}|}
\hline Name & Example & Description\\
\hline
\hline dataFile & foo.lst & Data to plot, one value by line in a text file\\
\hline nbBins & 10 & number of bins for the histogram, e.g.: Sturges advices $nbBins=[1+log_2(n)]$ where $n$ is the amount of data\\
\hline outFile & foo.hist & \\
\hline
\end{tabular}

\subsection{ReadFeatFile}
\subsubsection{Features}
Display values of a feature file on the standard output.

\subsubsection{Compulsory options}

\begin{tabular}{|c|c||p{8cm}|}
\hline Name & Example & Description\\
\hline
\hline file & ./test.prm & input feature file name\\
\hline type & SPRO3 & <SPRO3|SPRO4|RAW|HTK>, if RAW, vectSize has to be specified\\
\hline
\end{tabular}

\subsection{ReadModel}
\subsubsection{Features}
Display the values (mean, std, weights) of each distribution of a input mixture on the standard output.

\subsubsection{Compulsory options}

\begin{tabular}{|c|c||p{8cm}|}
\hline Name & Example & Description\\
\hline
\hline file & ./test.gmm & input mixture file name\\
\hline vectSize & 1024 & Assuming it's in RAW format\\
\hline
\end{tabular}

\subsection{Scoring}
\subsubsection{Features}
Scoring aims to take the decision accept/reject according to the confidence score found in the score file. Different scoring mode are provided for different tasks.
A NIST mode fitting the needs of the SRE 2004 evaluation.
A MDTM and a ETF mode fitting the needs of the ESTER 2005 evaluation, and more generally useful for speaker tracking scoring.
A \textit{--hard} option is provided to force a single ACCPET decision by test segment to be taken. The ACCEPT decision is made on the speaker having the highest score.
A \textit{--leaveMaxOutTnorm} option is provided to apply a pseudo-Tnorm normalization on scores. This is useful in closed-environment task. The TNorm is computed without the maximum score for each segment.

\subsubsection{Compulsory options}
\begin{tabular}{|c|c||p{8cm}|}
\hline Name & Example & Description\\
\hline
\hline mode &  NIST & \\Specify the type of scoring and output <NIST|ETF|MDTM|leaveMaxOutTnorm>
\hline inputFile & test1.res & input score file\\
\hline outputFile & test1.res.scored & normalized score file\\
\hline threshold & 2.4 & the threshold to take a decision\\
\hline hard & \textit{no value} & only one TRUE decision is taken by segment\\
\hline nbLoc & 20 & needed when \texit{--hard} or in the leaveMaxOutTnorm mode\\
\hline
\end{tabular}

\end{document}
